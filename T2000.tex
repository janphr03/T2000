\documentclass[12pt,a4paper]{article} % Standard für Hausarbeiten

% Sprachunterst\"utzung f\"ur deutsche Umlaute und Silbentrennung
\usepackage[utf8]{inputenc}  % Zeichencodierung
\usepackage[T1]{fontenc}     % Korrekte Darstellung von Umlauten
\usepackage[ngerman]{babel}  % Deutsche Sprache und Silbentrennung
\usepackage{csquotes}

% Seitenr\"ander sch\"oner machen
\usepackage[a4paper, left=3cm, right=3cm, top=2.5cm, bottom=2.5cm]{geometry}

% Mathematik-Symbole
\usepackage{amsmath, amssymb}

% Grafiken und Bilder einf\"ugen
\usepackage{graphicx}  
\usepackage{float}      % Bessere Kontrolle \über die Platzierung

% Tabellen verbessern
\usepackage{array, booktabs}

% Literaturverzeichnis mit BibTeX
\usepackage[style=ieee, backend=biber]{biblatex}
\addbibresource{literature.bib}
\addbibresource{Meine Bibliothek.bib}

% Quellcode schön darstellen
\usepackage{listings}
\usepackage{xcolor}

\usepackage{needspace} % in der Präambel

\usepackage[acronym]{glossaries}



\newacronym{loggerintf}{Logger-Interface}{Abstraktes Interface f\"ur Logging, das per DI eingespeist wird}




\makeglossaries



\lstset{ 
    language=C, % oder C++, Java, Python etc.
    basicstyle=\ttfamily\small, 
    keywordstyle=\color{blue}, 
    commentstyle=\color{gray}, 
    stringstyle=\color{red},
    breaklines=true
}

%Commands/ Style-Definitions====================================
\newcommand{\csharp}{C\#}

\setlength{\parskip}{1em}    % Abstand zwischen Abs\"atzen
\setlength{\parindent}{0pt}  % Kein Einzug bei neuen Abs\"atzen


% compile \"uber Konsole
% pdflatex test.tex
% biber test
% dann wieder pdflatex test.tex
% ba Datei Pfad anpassen
%T999-Beginn==================================================
\begin{document}


\title{T2000} 
\author{Jan Herrmann}
\date{\today}
\maketitle

%\section{Test}
%\Das ist meine LaTeX-Hausarbeit.
%\parencite{heimeshoff_certification_2024}
%testZitat
%\parencite{kim_planck_2024}


\tableofcontents
\newpage

%–– Im Dokument dann an der Stelle, an der du dein Abkürzungsverzeichnis haben willst ––
\printglossary[type=\acronymtype,title=Abkürzungsverzeichnis]



\section{Einleitung}

DEVELOP Test


\section{Theoretischer Hintergrund}
    

    \subsection{künstliche Intelligenz}
    Bevor eine präzise Definition von Künstlicher Intelligenz möglich ist, muss zunächst geklärt werden, welches Ziel KI-Systeme überhaupt verfolgen sollen. 
    Russell und Norvig zeigen, dass sich KI-Definitionen in wissenschaftlicher Literatur entlang zweier Dimensionen unterscheiden: 
    Zum einen kann der Fokus darauf liegen, ob ein System wie ein Mensch denkt oder handelt. 
    Zum anderen kann beurteilt werden, ob ein System rational denken oder rational handeln soll, 
    also ob sein Verhalten an einem idealen, logisch korrekten Maßstab orientiert wird. 
    Aus dieser Kombination ergeben sich vier grundlegende Perspektiven auf KI, 
    die historisch alle verfolgt wurden und bis heute unterschiedliche Forschungsrichtungen repräsentieren (vgl. \cite{russell_artificial_1995} S.~4). 
    
    Die folgende Tabelle fasst diese vier Sichtweisen zusammen.
    

        \begin{table}[h]
        \centering
        \begin{tabular}{|p{4cm}|p{9cm}|}
        \hline
        \textbf{Kategorie} & \textbf{Beschreibung} \\ \hline
        Systeme, die wie Menschen denken & Fokus auf Nachbildung menschlicher Denkprozesse, z.\,B. durch kognitive Modelle oder psychologische Theorien. \\ \hline
        Systeme, die wie Menschen handeln & Intelligenz wird anhand menschlich ähnlichen Verhaltens beurteilt, unabhängig vom zugrunde liegenden Denkprozess. \\ \hline
        Systeme, die rational denken & Fokus auf logische Schlussfolgerungen und formale Wissensrepräsentation. \\ \hline
        Systeme, die rational handeln & Intelligente Agenten handeln zielgerichtet und optimal in ihrer Umgebung. \\ \hline
        \end{tabular}
        \caption{Eigene Darstellung in Anlehnung an \cite{russell_artificial_1995}, S.~5.}
        \end{table}

        - Abgrenzung zu KI & ML



    \subsection{Grundlagen machine Learning}

        \subsubsection{Arten des Lernens}

        - überwacht

        - unüberwacht

        - bestärkend/ reinforcement

        - selbstüberwacht/ self-supervised

        \subsubsection{Daten& Features}

        - feature engineering vs feature learning

        - Datenqualität, BIAS, Overfitting, Underfitting



    \subsection{deep Learning}
        \subsubsection{Grundlagen neuronale Netze}

        - Neuron, Gewicht, BIAS

        - Aktivierungsfunktion

        \subsubsection{Training}
        - Loss Funktion

        - Konzept von Backpropagation

        - Optimierer ????

        \subsubsection{moderen Architekturen}

        - CNN, RNN, Transformer einordnen, evtl. noch andere raussuchen

    


\section{Vorgehensweise}
    \subsection{Zielsetzung}
    \subsection{Analyse}
    \subsection{Entwurf}
    \subsection{Umsetzung}
    \subsection{Validierung}


\section{Applikation}



\printbibliography  % Literaturverzeichnis einfügen



\end{document}


%Quellen