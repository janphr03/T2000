\documentclass[12pt,a4paper]{article} % Standard für Hausarbeiten

% Sprachunterst\"utzung f\"ur deutsche Umlaute und Silbentrennung
\usepackage[utf8]{inputenc}  % Zeichencodierung
\usepackage[T1]{fontenc}     % Korrekte Darstellung von Umlauten
\usepackage[ngerman]{babel}  % Deutsche Sprache und Silbentrennung
\usepackage{csquotes}

% Seitenr\"ander sch\"oner machen
\usepackage[a4paper, left=3cm, right=3cm, top=2.5cm, bottom=2.5cm]{geometry}

% Mathematik-Symbole
\usepackage{amsmath, amssymb}

% Grafiken und Bilder einf\"ugen
\usepackage{graphicx}  
\usepackage{float}      % Bessere Kontrolle \über die Platzierung

% Tabellen verbessern
\usepackage{array, booktabs}

% Literaturverzeichnis mit BibTeX
\usepackage[style=ieee, backend=biber]{biblatex}
\addbibresource{literature.bib}
\addbibresource{Meine Bibliothek.bib}

% Quellcode schön darstellen
\usepackage{listings}
\usepackage{xcolor}

\usepackage{needspace} % in der Präambel

\usepackage[acronym]{glossaries}



\newacronym{ki}{KI}{künstliche Intelligenz}
\newacronym{ml}{ML}{Machine learning}


\makeglossaries



\lstset{ 
    language=C, % oder C++, Java, Python etc.
    basicstyle=\ttfamily\small, 
    keywordstyle=\color{blue}, 
    commentstyle=\color{gray}, 
    stringstyle=\color{red},
    breaklines=true
}

%Commands/ Style-Definitions====================================
\newcommand{\csharp}{C\#}

\setlength{\parskip}{1em}    % Abstand zwischen Abs\"atzen
\setlength{\parindent}{0pt}  % Kein Einzug bei neuen Abs\"atzen


% compile \"uber Konsole
% pdflatex test.tex
% biber test
% dann wieder pdflatex test.tex
% ba Datei Pfad anpassen
%T999-Beginn==================================================
\begin{document}


\title{T2000} 
\author{Jan Herrmann}
\date{\today}
\maketitle

%\section{Test}
%\Das ist meine LaTeX-Hausarbeit.
%\parencite{heimeshoff_certification_2024}
%testZitat
%\parencite{kim_planck_2024}


\tableofcontents
\newpage

%–– Im Dokument dann an der Stelle, an der du dein Abkürzungsverzeichnis haben willst ––
\printglossary[type=\acronymtype,title=Abkürzungsverzeichnis]



\section{Einleitung}
    \subsection{Motivation}
    \subsection{Problemstellung}
    \subsection{Zielsetzung}

DEVELOP Test
\acrshort{ML}


\section{Theoretischer Hintergrund}
    


        \subsection{Künstliche Intelligenz}

            Bevor eine präzise Definition von künstlicher Intelligenz möglich ist, muss geklärt werden, welches Ziel ein intelligentes System verfolgen soll. 
            Russell und Norvig zeigen, dass gängige KI-Definitionen in der wissenschaftlichen Literatur entlang zweier zentraler Dimensionen variieren (vgl. \cite{russell_artificial_1995}, Kap.~1.1):

            \begin{itemize}
                \item \textbf{Mensch vs. Rationalität} Soll ein System wie ein Mensch denken oder handeln, oder soll es unabhängig vom Menschen ideal rational agieren?
                \item \textbf{Denken vs. Handeln} Soll Intelligenz anhand interner Denkprozesse oder anhand des beobachtbaren Verhaltens beurteilt werden?
            \end{itemize}

            Aus diesen beiden Dimensionen ergeben sich vier grundlegende Perspektiven auf KI, die unterschiedliche historische Forschungsrichtungen geprägt haben. 
            Eine Übersicht dieser Einordnung zeigt Tabelle~\ref{tab:ki-perspektiven}
        

            \begin{table}[h]
            \centering
            \begin{tabular}{|p{4cm}|p{9cm}|}
            \hline
            \textbf{Kategorie} & \textbf{Beschreibung} \\ \hline
            Systeme, die wie Menschen denken & Fokus auf Nachbildung menschlicher Denkprozesse, z.B. durch kognitive Modelle oder psychologische Theorien. \\ \hline
            Systeme, die wie Menschen handeln & Intelligenz wird anhand menschlich ähnlichen Verhaltens beurteilt, unabhängig vom zugrunde liegenden Denkprozess. \\ \hline
            Systeme, die rational denken & Fokus auf logische Schlussfolgerungen und formale Wissensrepräsentation. \\ \hline
            Systeme, die rational handeln & Intelligente Agenten handeln zielgerichtet und optimal in ihrer Umgebung. \\ \hline
            \end{tabular}
            \caption{Eigene Darstellung in Anlehnung an \cite{russell_artificial_1995}, Kap.~1.1}
            \label{tab:ki-perspektiven}
            \end{table}


            \textbf{Abgrenzung von KI und ML}

            Künstliche Intelligenz (\acrshort{ki}) umfasst alle Verfahren, die darauf abzielen,
            intelligentes Verhalten technisch zu realisieren. Dazu gehören sowohl symbolische
            Ansätze wie Wissensrepräsentation und logisches Schließen als auch
            datengetriebene Methoden zur Wahrnehmung oder Sprachverarbeitung
            (vgl. \cite{russell_artificial_1995}, Kap.~1.1).

            Maschinelles Lernen (\acrshort{ml}) stellt ein klar abgegrenztes Teilgebiet der KI dar.
            Russell und Norvig beschreiben es als das Teilfeld der künstlichen Intelligenz,
            das sich mit Programmen befasst, die aus Erfahrung lernen
            (vgl. \cite{russell_artificial_1995}, Einleitung zu Teil~VI).
            

            Während KI somit als Oberbegriff sämtliche Methoden intelligenter Problemlösung
            einschließt, konzentriert sich ML ausschließlich auf Verfahren, die Wissen nicht
            explizit vorgegeben bekommen, sondern selbstständig aus Daten oder Erfahrungen
            erschließen. ML bildet damit die Grundlage für viele moderne KI-Anwendungen,
            insbesondere für datengetriebene Systeme wie neuronale Netze oder Large Language Models.
            


        

    \subsection{Vektorstore}
            Vektorstore ist nur ein Konzept, welche Unterschiede gibt es da?
            Konzept: "Eine DB die Embeddings speichert und semantische Ähnlichkeitssuche durchführen kann"
            
            Vektor Store vs Vektor Datenbank
            \textbf{Vektor Store}

            \textbf{}


            
    \subsection{Neuronale Netze}
    \subsection{Large Language Models}
    \subsection{Retrieval-Augmented Generation} % LLM mit Datenquelle
    \subsection{Schnittstellentechnologie}
    \subsection{prompt Engineering}


\section{Anforderungsanalyse}
    \subsection{Zielgruppenanalyse}
    \subsection{Benötigte Daten aus dem PIM System}
    \subsection{Vergleich der LLM Modelle}


\section{Konzeption und Realisierung}
    \subsection{Architektur}
    \subsection{Datenfluss zwischen Vektor Store und Applikation}
    \subsection{Schnittstellendesign}
    \subsection{Promptdesign}


\section{Implementierung des Prototyps}
    \subsection{Überblick über die Systemkomponenten}
    \subsection{Umsetzung der Schnittstellen (Vector Store / Applikation / LLM)}
    \subsection{Datenimport und -export}
    \subsection{Integration des LLMs und Promptlogik}
    \subsection{Fehlerbehandlung und Parallelität}


\section{Evaluation}
        \subsection{Aufbau der Evaluationsbewertung}
        \subsection{Bewertungsmetrik/ -kriterien} %
        \subsection{Durchführung}
        Wer bewertet Texte, Bewertungsbogen, Wie viele Bewertungen pro Text 

        \subsection{}
    
        

\printbibliography  % Literaturverzeichnis einfügen



\end{document}


%Quellen