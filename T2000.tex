\documentclass[12pt,a4paper]{article} % Standard für Hausarbeiten

% Sprachunterst\"utzung f\"ur deutsche Umlaute und Silbentrennung
\usepackage[utf8]{inputenc}  % Zeichencodierung
\usepackage[T1]{fontenc}     % Korrekte Darstellung von Umlauten
\usepackage[ngerman]{babel}  % Deutsche Sprache und Silbentrennung
\usepackage{csquotes}

% Seitenr\"ander sch\"oner machen
\usepackage[a4paper, left=3cm, right=3cm, top=2.5cm, bottom=2.5cm]{geometry}

% Mathematik-Symbole
\usepackage{amsmath, amssymb}

% Grafiken und Bilder einf\"ugen
\usepackage{graphicx}  
\usepackage{float}      % Bessere Kontrolle \über die Platzierung

% Tabellen verbessern
\usepackage{array, booktabs}

% Literaturverzeichnis mit BibTeX
\usepackage[style=ieee, backend=biber]{biblatex}
\addbibresource{literature.bib}
\addbibresource{Meine Bibliothek.bib}

% Quellcode schön darstellen
\usepackage{listings}
\usepackage{xcolor}

\usepackage{needspace} % in der Präambel

\usepackage[acronym]{glossaries}



\newacronym{loggerintf}{Logger-Interface}{Abstraktes Interface f\"ur Logging, das per DI eingespeist wird}




\makeglossaries



\lstset{ 
    language=C, % oder C++, Java, Python etc.
    basicstyle=\ttfamily\small, 
    keywordstyle=\color{blue}, 
    commentstyle=\color{gray}, 
    stringstyle=\color{red},
    breaklines=true
}

%Commands/ Style-Definitions====================================
\newcommand{\csharp}{C\#}

\setlength{\parskip}{1em}    % Abstand zwischen Abs\"atzen
\setlength{\parindent}{0pt}  % Kein Einzug bei neuen Abs\"atzen


% compile \"uber Konsole
% pdflatex test.tex
% biber test
% dann wieder pdflatex test.tex
% ba Datei Pfad anpassen
%T999-Beginn==================================================
\begin{document}


\title{T2000} 
\author{Jan Herrmann}
\date{\today}
\maketitle

%\section{Test}
%\Das ist meine LaTeX-Hausarbeit.
%\parencite{heimeshoff_certification_2024}
%testZitat
%\parencite{kim_planck_2024}


\tableofcontents
\newpage

%–– Im Dokument dann an der Stelle, an der du dein Abkürzungsverzeichnis haben willst ––
\printglossary[type=\acronymtype,title=Abkürzungsverzeichnis]



\section{Einleitung}

DEVELOP Test

\section{Theoretischer Hintergrund}
    
    \subsection{Einführung in KI}
    
        ML vs DL vs KI


    \subsection{künstliche Intelligenz}

        - Definition

        - Abgrenzung zu KI& ML


    \subsection{Grundlagen machine Learning}

        \subsubsection{Arten des Lernens}
        - überwacht

        - unüberwacht

        - bestärkend/ reinforcement

        - selbstüberwacht/ self-supervised

        \subsubsection{Daten& Features}

        - feature engineering vs feature learning

        - Datenqualität, BIAS, Overfitting, Underfitting

    \subsection{deep Learning}
        \subsubsection{Grundlagen neuronale Netze}

        - Neuron, Gewicht, BIAS

        - Aktivierungsfunktion

        \subsubsection{Training}
        - Loss Funktion

        - Konzept von Backpropagation

        - Optimierer ????

        \subsubsection{moderen Architekturen}

        - CNN, RNN, Transformer einordnen, evtl. noch andere raussuchen

    


\section{Vorgehensweise}
    \subsection{Zielsetzung}
    \subsection{Analyse}
    \subsection{Entwurf}
    \subsection{Umsetzung}
    \subsection{Validierung}

\section{Applikation}



\printbibliography  % Literaturverzeichnis einfügen



\end{document}


%Quellen